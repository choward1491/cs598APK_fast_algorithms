\documentclass{article}[11pt]
\usepackage{amsmath}
\usepackage{amssymb}
\usepackage{cancel}
\usepackage{ntheorem}
\renewcommand{\qedsymbol}{\rule{0.7em}{0.7em}}

\newtheorem{lemma}{Lemma}[section]
\newtheorem{result}{Result}[section]

\makeatletter
\newcommand{\chapterauthor}[1]{%
  {\parindent0pt\vspace*{-25pt}%
  \linespread{1.1}\large\scshape#1%
  \par\nobreak\vspace*{35pt}}
  \@afterheading%
}
\makeatother

%\title{Homework 1}
\author{Christian Howard}
%\date{\today}




\begin{document}
   %\maketitle
   
   \section{Wrapping Many Integrations into One}
   \textbf{Author}: Christian Howard \\
   Using the problem statement, we can define the following integral operators:
   
   \begin{align}
   D^{-1} \phi (x) &= \int_0^x \phi(y) dy \\
   D^{-n} \phi (x) &= \frac{1}{(n-1)!}\int_0^x (x-y)^{n-1}\phi(y) dy \label{eqn}
   \end{align}
   
   Given the above operators, the goal is to prove $D^{-n} \phi (x)$ is correct for all positive integers $n$. To do this, let us first recall the following based on previous work:
   
   \begin{lemma} \label{l1}
   Given two integral operators, $F\phi(x)$ and $G\phi(x)$, and their corresponding kernels, $k_1(\cdot,\cdot)$ and $k_2(\cdot,\cdot)$, the kernel $k_3(\cdot,\cdot)$ within $(G \circ F)\phi(x)$ can be found using:
	   \begin{align*}
	   k_3(x,y) = \int_y^x k_2(x,z) k_1(z,y) dz 
	   \end{align*}
   \end{lemma}
   
   To proceed in proving (\ref{eqn}) is correct, we can first check that (\ref{eqn}) satisfies the base case, where $n=1$, by doing the following:
   
   \begin{align*}
   \left. D^{-n} \phi (x)\right|_{n=1} &= \frac{1}{(1-1)!}\int_0^x (x-y)^{1-1}\phi(y) dy \\
   &= \int_0^x \phi(y) dy 
   \end{align*}
   
   Now let us assume that (\ref{eqn}) holds for $0 \leq n \leq k$. We can then find $D^{-(k+1)}\phi(x)$ by first noting the following relationship:
   \begin{align*}
   D^{-(k+1)}\phi(x) &= (D^{-1} \circ D^{-k})\phi(x) = \int_0^x K(x,y) \phi(y) dy
   \end{align*}
   
   Using our inductive hypothesis that $D^{-k}\phi(x)$ holds and Lemma \ref{l1}, we can find the resulting kernel, $K(\cdot,\cdot)$, for $(D^{-1} \circ D^{-k})\phi(x)$ to be the following:
   
   \begin{align*}
   K(x,y) &= \int_y^x \frac{(x-z)^{k-1}}{(k-1)!} dz \\
   &= \left( -\frac{(x-z)^{k}}{k(k-1)!} \right]_y^x \\
   &= \frac{(x-y)^{k}}{k!}
   \end{align*}
   
   With the above kernel, we can find the final form for $D^{-(k+1)}\phi(x)$ to be:
   \begin{align*}
   D^{-(k+1)}\phi(x) &= \frac{1}{k!}\int_0^x (x-y)^{k} \phi(y) dy
   \end{align*}
   
   Thus, the form of $D^{-(k+1)}\phi(x)$ matches (\ref{eqn}) when $n=k+1$, completing the induction step. Now by the principle of induction, (\ref{eqn}) holds $\forall n \in \mathbb{N}^{+}$.
   
\end{document}