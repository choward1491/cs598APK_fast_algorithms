\documentclass{article}[11pt]
\usepackage{amsmath}
\usepackage{amssymb}
\usepackage{amsthm}
\usepackage{cancel}
\renewcommand{\qedsymbol}{\rule{0.7em}{0.7em}}

\newtheorem{lemma}{Result}[section]

\makeatletter
\newcommand{\chapterauthor}[1]{%
  {\parindent0pt\vspace*{-25pt}%
  \linespread{1.1}\large\scshape#1%
  \par\nobreak\vspace*{35pt}}
  \@afterheading%
}
\makeatother

%\title{Homework 1}
\author{Christian Howard}
%\date{\today}




\begin{document}
   %\maketitle
   
   \section{Iterated Integral Operators}
   \textbf{Author}: Christian Howard \\
   In the problem statement, we can define the following integral operators:
   
   \begin{align}
   F \phi (x) &= \int_a^x k_1(x,y) \phi(y) dy \\
   G \phi (x) &= \int_a^x k_2(x,y) \phi(y) dy 
   \end{align}
   
   Since we are relating these integral operators to upper/triangular matrices, I make the following assumption based on the integral operator bounds:
   
   \begin{align}
   k_1(x,y) = k_2(x,y) = 0 \;\;\forall y > x
   \end{align}
   
   Given the above operators and properties, the goal is to find $k_3(x,y)$ $\ni$
   
   \begin{align}
   G(F\phi)(x) = (G \circ F)\phi(x) = \int_a^x k_3(x,y) \phi(y) dy
   \end{align}
   
   
   \begin{lemma}
   The resulting kernel within $(G \circ F)\phi(x)$ is:
	   \begin{align*}
	   k_3(x,y) = \int_y^x k_2(x,z) k_1(z,y) dz 
	   \end{align*}
   \end{lemma}
   
   \begin{proof}
   \begin{align*}
   (G \circ F)\phi(x) &= \int_a^x k_2(x,z) F\phi(z) dz \\
   &= \int_a^x k_2(x,z) \int_a^z k_1(z,y) \phi(y) dy dz \\
   &= \int_a^x \phi(y) \int_a^x k_2(x,z) k_1(z,y)  dz dy - \int_a^x k_2(x,z) \cancelto{0}{\int_z^x k_1(z,y) \phi(y) dy} dz \\
   &= \int_a^x \phi(y) \cancelto{0}{\int_a^y k_2(x,z) k_1(z,y)  dz} dy + \int_a^x \phi(y) \int_y^x k_2(x,z) k_1(z,y)  dz dy \\
   &=  \int_a^x \phi(y) \int_y^x k_2(x,z) k_1(z,y)  dz dy \\
   &=  \int_a^x k_3(x,y) \phi(y) dy \\
   \therefore k_3(x,y) &= \int_y^x k_2(x,z) k_1(z,y) dz
   \end{align*}
   \end{proof}
   
\end{document}