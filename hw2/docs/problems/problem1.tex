\documentclass{article}[11pt]
\usepackage{amsmath}
\usepackage{amssymb}
\usepackage{amsthm}
\usepackage{cancel}
\usepackage{enumitem}
\renewcommand{\qedsymbol}{\rule{0.7em}{0.7em}}
%\renewcommand{\theenumi}{\Alph{enumi}.}

\usepackage[utf8]{inputenc}
\usepackage[english]{babel}
%\usepackage[nottoc]{tocbibind} %Adds "References" to the table of contents

\newtheorem{lemma}{Result}[section]

\makeatletter
\newcommand{\chapterauthor}[1]{%
  {\parindent0pt\vspace*{-25pt}%
  \linespread{1.1}\large\scshape#1%
  \par\nobreak\vspace*{35pt}}
  \@afterheading%
}
\makeatother

\title{Homework 2 \\ Problem 1 - Working with Low-Rank Approximations}
\author{Christian Howard \\ howard28@illinois.edu }
\date{}




\begin{document}
   \maketitle
   
   \newpage
   
   \tableofcontents
   
   \newpage
   
   \section{Algorithm Design}
   In the problem statement for this part, we are to come up with an algorithm to tackle least square problems $Ax \cong b$ given that $A \in \mathbb{R}^{m \times n}$ is low rank and approximated as $A \approx QB$, where $B = Q^{T}A$ with $m > n \geq k$ and $Q \in \mathbb{R}^{m \times k}$. Additionally, for cases where the least square solution is not unique, i.e. when $k < n$, we are asked to add that the solution $x$ will minimize $|x|_2$. 
   
   If we first assume that $k=n$, we can find the least square solution by finding $x^* = \arg \min_{x} J(x)$ where $J(x)$ is defined below as:
   
   \begin{align*}
   J(x) &= (Ax - b)^T (Ax-b) \\
      &= x^TA^TAx - 2x^T A^T b + b^Tb
   \end{align*}   
   
   If we differentiate with respect to $x^T$, we can find the solution by doing the following:
   
   \begin{align*}
   \frac{\partial J}{\partial x^T} = 0 &= 2A^TAx^* - 2A^T b \\
   &= A^TAx^* - A^T b 
   \end{align*}
   
   which implies a solution of:
   
   \begin{align*}
   x^* = \left(A^T A\right)^{-1} A^Tb
   \end{align*}    
   
  If we assume $A$ is in the form of a Singular Value Decomposition, i.e. $A = U\Sigma V^T$, then we can show the following solution for the Least Square solution when $k = n$:
  
     \begin{align}
   x^* &= \left(V \Sigma U^T U \Sigma V^T\right)^{-1} V \Sigma U^T b \nonumber\\
   &= \left(V \Sigma^2 V^T\right)^{-1} V \Sigma U^T b \nonumber\\
   &= V \Sigma^{-2} V^T V \Sigma U^T b \nonumber\\
   &= V \Sigma^{-1} U^T b \nonumber \\
   &= V \Sigma^{+} U^T b \label{eq:ls1}
   \end{align}    
   
   where $\Sigma^{+}$ is the pseudoinverse of $\Sigma$ where basically all non-zero diagonal elements are replaced by their reciprocals. Now let us assume that $k < n$ so that we need to enforce that the solution $x^*$ must also minimize $|x|_2$. We can find this solution $x^*$ by solving the following optimization problem based on Lagrange Multipliers:
   
   \begin{align*}
   (x^*,\lambda^*) &= \arg \min_{x,\lambda} J(x,\lambda) \\
   \text{where } J(x,\lambda) &= x^Tx + \lambda^T\left(Ax - b\right)
   \end{align*}
   
    If we differentiate $J(\cdot,\cdot)$ with respect to $x^T$ and $\lambda^T$, we get:
    
    \begin{align*}
    \frac{\partial J}{\partial x^T} = 0 &= 2x^* + A^T\lambda^* \\
    \frac{\partial J}{\partial \lambda^T} = 0 &= Ax^* - b
    \end{align*}
    
    The first equation implies $x^* = -\frac{1}{2} A^T \lambda$, the second leads to $\lambda^* = -2 (A A^T)^{-1}b$. With these relationships, we find the solution for $x^*$ is the following:
    
    \begin{align*}
    x^* &= A^T \left(A A^T\right)^{-1}b
    \end{align*}
    
    If we again substitute the SVD form of $A$ into the expression for $x^*$, we can obtain the following:
    \begin{align}
   x^* &= V \Sigma U^T \left( U \Sigma V^T V \Sigma U^T \right)^{-1} b \nonumber  \\
   &=V \Sigma U^T \left( U \Sigma^2 U^T \right)^{-1} b \nonumber \\
   &= V \Sigma U^T U \Sigma^{-2} U^T  b \nonumber \\
   &= V \Sigma^{-1} U^T  b \nonumber \\
   &= V \Sigma^{+} U^T  b \label{eq:ls2}
   \end{align}    
    
    What becomes obvious is that if we compare (\ref{eq:ls1}) with (\ref{eq:ls2}), the two solutions are identical. This means that using the SVD form of $A$ to solve the Least Square problem works when $k \leq n$ and enforces minimizing $|x|_2$ when $k < n$. With this information, the algorithms we come up with only need to use the information we have to construct efficient SVD decompositions and then the rest is trivial.
   
   \subsection{Algorithm 1 - Baseline}
   \begin{enumerate}
   \item Find SVD of $A = QB$
   		\begin{enumerate}
   		\item Find SVD of $B \ni B = \hat{U}\Sigma V^T$
   		\item Define $U = Q\hat{U} \ni A = U \Sigma V^T$ 
   		\end{enumerate}
   \item Compute $x^* = V \Sigma^{+} U^T b$
   \end{enumerate}
   
   \subsection{Algorithm 2 - Interpolative Decomposition}
   \begin{enumerate}
   \item Find SVD of $A = QB$
   		\begin{enumerate}
   		\item Find $J$ and $P$ using an Interpolative Decomposition $\ni Q^{T} = Q^{T}_{(:,J)}P^T$
   		\item Compute QR factorization $(A_{(J,:)})^T = \bar{Q}\bar{R}$
   		\item Upsample row coefficients, $Z = P\bar{R}^T$
   		\item Compute SVD of $Z \ni Z = U \Sigma \hat{V}^T$
   		\item Define $V^T = \hat{V}^T \bar{Q}^T \ni A = U \Sigma V^T$
   		\end{enumerate}
   \item Compute $x^* = V \Sigma^{+} U^T b$
   \end{enumerate}
   
   \newpage
   \section{Computational Complexity}
   Before going into the complexity analysis, I want to first note that \cite{labook} states that a Full Householder QR factorization of some matrix $A \in \mathbb{R}^{m \times n}$ is $O(m^2n)$. Additionally, I will use the $n$-Truncated R-SVD such that for some matrix $A \in \mathbb{R}^{m \times n}$ where $m \geq n$, the complexity is $O(mn^2)$. For use in analyzing the Interpolative Decomposition (ID), most of the heavy lifting of ID is based on the Column Pivoted QR Algorithm with complexity $O(mnk)$ for a matrix $A \in \mathbb{R}^{m \times n}$ that is rank $k$. These complexities will be used to describe the algorithms below.
   
   \subsection{Algorithm 1 - Baseline}
   \begin{enumerate}
   \item Find SVD of $A = QB$ for $A \in \mathbb{R}^{m \times n}$, $Q \in \mathbb{R}^{m \times k}$, and $B \in \mathbb{R}^{k \times n}$
   		\begin{enumerate}
   		\item Find SVD of $B \ni B = \hat{U}\Sigma V^T$. Note that $\hat{U} \in \mathbb{R}^{k \times k}$, $\Sigma \in \mathbb{R}^{k \times k}$, and $V \in \mathbb{R}^{n \times k}$
   			\begin{itemize}
   			\item Complexity: $O(nk^2)$
   			\end{itemize}
   		\item Define $U = Q\hat{U} \ni A = U \Sigma V^T$ 
   			\begin{itemize}
   			\item Complexity: $O(m k^2)$
   			\end{itemize}
   		\end{enumerate}
   \item Compute $x^* = V \Sigma^{+} U^T b$
   		\begin{enumerate}
   		\item Compute $\Sigma^{+}$ from $\Sigma$
   			\begin{itemize}
   			\item Complexity: $O(k)$
   			\end{itemize}
   		\item Define $q_1 = U^{T}b$ 
   			\begin{itemize}
   			\item Complexity: $O(km)$
   			\end{itemize}
   		\item Define $q_2 = \Sigma^{+}q_1$ 
   			\begin{itemize}
   			\item Complexity: $O(k^2)$
   			\end{itemize}
   		\item Define $x^* = Vq_2$ 
   			\begin{itemize}
   			\item Complexity: $O(kn)$
   			\end{itemize}
   		\end{enumerate}
   \end{enumerate}
   
   The overall complexity of Algorithm 1 as $O(n k^2 + m k^2 + km) \in O(m k^2)$.
   
   \subsection{Algorithm 2 - Interpolative Decomposition}
   \begin{enumerate}
   \item Find SVD of $A = QB$
   		\begin{enumerate}
   		\item Find $J$ and $P$ using an Interpolative Decomposition $\ni Q^{T} = Q^{T}_{(:,J)}P^T$
   			\begin{itemize}
   			\item Complexity: $O(m k^2)$
   			\end{itemize}
   		\item Compute QR factorization $(A_{(J,:)})^T = \bar{Q}\bar{R}$
   			\begin{itemize}
   			\item Compute $C = (A_{(J,:)})(A_{(J,:)})^T \rightarrow O(mk^2)$
   			\item Compute $\bar{R} = \text{cholesky}(C) \rightarrow O(k^3)$
   			\item Compute $\bar{R}^{-1}$ $\rightarrow O(k^3)$
   			\item Compute $\bar{Q} = (A_{(J,:)})^T\bar{R}^{-1} \rightarrow O(mk^2)$
   			\end{itemize}
   		\item Upsample row coefficients, $Z = P\bar{R}^T$
   			\begin{itemize}
   			\item Complexity: $O(m k^2 )$
   			\end{itemize}
   		\item Compute SVD of $Z \in \mathbb{R}^{m \times k} \ni Z = U \Sigma \hat{V}^T$
   			\begin{itemize}
   			\item Complexity: $O(m k^2)$
   			\end{itemize}
   		\item Define $V^T = \hat{V}^T \bar{Q}^T \ni A = U \Sigma V^T$
   			\begin{itemize}
   			\item Complexity: $O(m k^2)$
   			\end{itemize}
   		\end{enumerate}
   \item Compute $x^* = V \Sigma^{+} U^T b \rightarrow O(km)$
   \end{enumerate}
   
   The overall complexity of Algorithm 2 is $O(mk^2)$.
   
\subsection{Discussion of Algorithms}
So what is interesting to note between the algorithms is they are both of complexity $O(m n^2)$ when $k = n$ since both could benefit from the Truncated R-SVD algorithm. This appears to be a different result than what is talked about in lecture since it is assumed the SVD of $B \in \mathbb{R}^{k \times n}$ in Algorithm 1 should be $O(n^2k)$, but if we can construct the SVD of $Z \in \mathbb{R}^{m \times k}$  in $O(mk^2)$, then we should be able to also construct the SVD of $B$ in $O(nk^2)$. This obviously goes against what is mentioned in the lecture notes, so not sure where the disparity lies.
   
   \newpage
   \section{ Complexity of Finding Low-rank Projection Matrix }
   \subsection{Nominal Non-Adaptive Range Finder}
   \begin{enumerate}
   \item Construct random matrix $\Omega \in \mathbb{R}^{n \times k} \rightarrow O(nk)$
   \item Get measurements from $A$ by doing $Y = A\Omega \rightarrow O(mnk)$
   \item Perform $l$ iterations of the following based on the Power Iteration:
   		\begin{itemize}
   		\item Perform QR of $Y \ni Y = \hat{Q} \hat{R} \rightarrow O(m^2k)$
   		\item Perform QR of $T = A^T \hat{Q}_{(:,1:k)}  \ni T = \bar{Q}\bar{R} \rightarrow O(mnk + n^2k)$
   		\item Perform $Y = A \bar{Q}_{(:,1:k)} \rightarrow O(nmk)$
   		\end{itemize}
   \item Perform QR of $Y \ni Y = \hat{Q} \hat{R} \rightarrow O(m^2k)$
   \item Return $Q = \hat{Q}_{(:,1:k)} \rightarrow O(mk)$
   \end{enumerate}
   
   The overall complexity of the Non-adaptive Range Finder is $O(m^2 k)$ since this is the dominant complexity term throughout the algorithm.
   
   \subsection{Subsampled Random Fourier Transform Non-Adaptive Range Finder}
   This SRFT Range Finder is based on algorithms described in \cite{fdft} that allow one to speed up the nominal FFT and inverse FFT computations when you only care about either a subset of inputs points or a subset of output points. That said, I do not see how we can use this to speed up the computation of $Y = A\Omega^{'}$, where $\Omega^{'}$ is defined as:
   
   \begin{align*}
   \Omega^{'} &= \sqrt{\frac{n}{k}} D F R
   \end{align*}
   
   where 
   
   \begin{itemize}
   \item $D$ is an $n \times n$ diagonal matrix whose entries are independent random variables uniformly distributed on the complex unit circle
   \item $F$ is the unitary $n \times n$ DFT matrix based on the relationship \\ $$F_{pq} = \frac{1}{\sqrt{n}} e^{-2\pi i (p-1) (q-1) / n}$$
   \item $R$ is an $n\times k$ matrix that samples $k$ coordinates from $n$ uniformly at random, i.e., its $k$ columns are drawn randomly without replacement from the columns of the $n\times n$ identity matrix
   \end{itemize}
   
   \newpage
   \section{Prove One Statement is True}
   We are given that $C$ is a square, low-rank matrix. Given this information, show that exactly one of the below statements are true:
   
   \begin{enumerate}[label=(\Alph*)]
   \item The linear system $(I-C)x = b$ has a solution $x$
   \item The linear system $(I-C)^Ty = 0$ has a solution $y$ such that $y^Tb \neq 0$
   \end{enumerate}
   
   For only one of the above statements to be true, we must show $A \iff \neg B$. To do this, we must show that $A \rightarrow \neg B$ and $\neg B \rightarrow A$. To prove $A \rightarrow \neg B$, let us assume $A$ is true and that $(I-C)^Ty = 0$ for some $y$. Then we have:
   
   \begin{align*}
   (I-C)x &= b \\
   y^T(I - C)x &= y^Tb \\
   0 &= y^Tb
   \end{align*}
   
   Since $y^Tb = 0$, $B$ cannot be true when $A$ is true, implying that $A \rightarrow \neg B$. Now let us assume $\neg A$ and that $(I-C)^Ty = 0$ for some $y$. Then we have:
   
   \begin{align*}
   (I-C)x &\neq b \\
   y^T(I-C)x & \neq y^Tb  \\
   0 &\neq  y^Tb
   \end{align*}
   
   The above result implies $\neg  A \rightarrow B$ since both $(I-C)^Ty = 0$ and $y^Tb \neq 0$ were satisfied. Since $\neg A \rightarrow B$ is the contrapositive of $\neg B \rightarrow A$, we have now shown $ A \iff \neg B$. This proves that only $A$ or $B$ can be true for some low-rank matrix $C$.
   
   
   
\begin{thebibliography}{9}
\bibitem{labook} 
Gene H. Golub, Charles F. Van Loan. 
\textit{Matrix Computations 4\textsuperscript{th} Edition}. 
The John Hopkins University Press, Baltimore, Maryland, 2013.
 
\bibitem{fdft} 
Henrik V. Sorensen, C. Sidney Burrus.
\textit{Efficient Computation of the DFT with Only a Subset of Input or Output Points}.
IEEE Transactions on Signal Processing, Volume: 41, Issue: 3, pgs. 1184 - 1200, Mar 1993.
\end{thebibliography}

\end{document}